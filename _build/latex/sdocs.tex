% Generated by Sphinx.
\def\sphinxdocclass{report}
\documentclass[letterpaper,10pt,english]{sphinxmanual}
\usepackage[utf8]{inputenc}
\DeclareUnicodeCharacter{00A0}{\nobreakspace}
\usepackage{cmap}
\usepackage[T1]{fontenc}
\usepackage{babel}
\usepackage{times}
\usepackage[Bjarne]{fncychap}
\usepackage{longtable}
\usepackage{sphinx}
\usepackage{multirow}


\title{sdocs}
\date{December 02, 2014}
\release{0.1.alpha}
\author{Jeffrey Scherling}
\newcommand{\sphinxlogo}{}
\renewcommand{\releasename}{Release}
\makeindex

\makeatletter
\def\PYG@reset{\let\PYG@it=\relax \let\PYG@bf=\relax%
    \let\PYG@ul=\relax \let\PYG@tc=\relax%
    \let\PYG@bc=\relax \let\PYG@ff=\relax}
\def\PYG@tok#1{\csname PYG@tok@#1\endcsname}
\def\PYG@toks#1+{\ifx\relax#1\empty\else%
    \PYG@tok{#1}\expandafter\PYG@toks\fi}
\def\PYG@do#1{\PYG@bc{\PYG@tc{\PYG@ul{%
    \PYG@it{\PYG@bf{\PYG@ff{#1}}}}}}}
\def\PYG#1#2{\PYG@reset\PYG@toks#1+\relax+\PYG@do{#2}}

\expandafter\def\csname PYG@tok@gd\endcsname{\def\PYG@tc##1{\textcolor[rgb]{0.63,0.00,0.00}{##1}}}
\expandafter\def\csname PYG@tok@gu\endcsname{\let\PYG@bf=\textbf\def\PYG@tc##1{\textcolor[rgb]{0.50,0.00,0.50}{##1}}}
\expandafter\def\csname PYG@tok@gt\endcsname{\def\PYG@tc##1{\textcolor[rgb]{0.00,0.27,0.87}{##1}}}
\expandafter\def\csname PYG@tok@gs\endcsname{\let\PYG@bf=\textbf}
\expandafter\def\csname PYG@tok@gr\endcsname{\def\PYG@tc##1{\textcolor[rgb]{1.00,0.00,0.00}{##1}}}
\expandafter\def\csname PYG@tok@cm\endcsname{\let\PYG@it=\textit\def\PYG@tc##1{\textcolor[rgb]{0.25,0.50,0.56}{##1}}}
\expandafter\def\csname PYG@tok@vg\endcsname{\def\PYG@tc##1{\textcolor[rgb]{0.73,0.38,0.84}{##1}}}
\expandafter\def\csname PYG@tok@m\endcsname{\def\PYG@tc##1{\textcolor[rgb]{0.13,0.50,0.31}{##1}}}
\expandafter\def\csname PYG@tok@mh\endcsname{\def\PYG@tc##1{\textcolor[rgb]{0.13,0.50,0.31}{##1}}}
\expandafter\def\csname PYG@tok@cs\endcsname{\def\PYG@tc##1{\textcolor[rgb]{0.25,0.50,0.56}{##1}}\def\PYG@bc##1{\setlength{\fboxsep}{0pt}\colorbox[rgb]{1.00,0.94,0.94}{\strut ##1}}}
\expandafter\def\csname PYG@tok@ge\endcsname{\let\PYG@it=\textit}
\expandafter\def\csname PYG@tok@vc\endcsname{\def\PYG@tc##1{\textcolor[rgb]{0.73,0.38,0.84}{##1}}}
\expandafter\def\csname PYG@tok@il\endcsname{\def\PYG@tc##1{\textcolor[rgb]{0.13,0.50,0.31}{##1}}}
\expandafter\def\csname PYG@tok@go\endcsname{\def\PYG@tc##1{\textcolor[rgb]{0.20,0.20,0.20}{##1}}}
\expandafter\def\csname PYG@tok@cp\endcsname{\def\PYG@tc##1{\textcolor[rgb]{0.00,0.44,0.13}{##1}}}
\expandafter\def\csname PYG@tok@gi\endcsname{\def\PYG@tc##1{\textcolor[rgb]{0.00,0.63,0.00}{##1}}}
\expandafter\def\csname PYG@tok@gh\endcsname{\let\PYG@bf=\textbf\def\PYG@tc##1{\textcolor[rgb]{0.00,0.00,0.50}{##1}}}
\expandafter\def\csname PYG@tok@ni\endcsname{\let\PYG@bf=\textbf\def\PYG@tc##1{\textcolor[rgb]{0.84,0.33,0.22}{##1}}}
\expandafter\def\csname PYG@tok@nl\endcsname{\let\PYG@bf=\textbf\def\PYG@tc##1{\textcolor[rgb]{0.00,0.13,0.44}{##1}}}
\expandafter\def\csname PYG@tok@nn\endcsname{\let\PYG@bf=\textbf\def\PYG@tc##1{\textcolor[rgb]{0.05,0.52,0.71}{##1}}}
\expandafter\def\csname PYG@tok@no\endcsname{\def\PYG@tc##1{\textcolor[rgb]{0.38,0.68,0.84}{##1}}}
\expandafter\def\csname PYG@tok@na\endcsname{\def\PYG@tc##1{\textcolor[rgb]{0.25,0.44,0.63}{##1}}}
\expandafter\def\csname PYG@tok@nb\endcsname{\def\PYG@tc##1{\textcolor[rgb]{0.00,0.44,0.13}{##1}}}
\expandafter\def\csname PYG@tok@nc\endcsname{\let\PYG@bf=\textbf\def\PYG@tc##1{\textcolor[rgb]{0.05,0.52,0.71}{##1}}}
\expandafter\def\csname PYG@tok@nd\endcsname{\let\PYG@bf=\textbf\def\PYG@tc##1{\textcolor[rgb]{0.33,0.33,0.33}{##1}}}
\expandafter\def\csname PYG@tok@ne\endcsname{\def\PYG@tc##1{\textcolor[rgb]{0.00,0.44,0.13}{##1}}}
\expandafter\def\csname PYG@tok@nf\endcsname{\def\PYG@tc##1{\textcolor[rgb]{0.02,0.16,0.49}{##1}}}
\expandafter\def\csname PYG@tok@si\endcsname{\let\PYG@it=\textit\def\PYG@tc##1{\textcolor[rgb]{0.44,0.63,0.82}{##1}}}
\expandafter\def\csname PYG@tok@s2\endcsname{\def\PYG@tc##1{\textcolor[rgb]{0.25,0.44,0.63}{##1}}}
\expandafter\def\csname PYG@tok@vi\endcsname{\def\PYG@tc##1{\textcolor[rgb]{0.73,0.38,0.84}{##1}}}
\expandafter\def\csname PYG@tok@nt\endcsname{\let\PYG@bf=\textbf\def\PYG@tc##1{\textcolor[rgb]{0.02,0.16,0.45}{##1}}}
\expandafter\def\csname PYG@tok@nv\endcsname{\def\PYG@tc##1{\textcolor[rgb]{0.73,0.38,0.84}{##1}}}
\expandafter\def\csname PYG@tok@s1\endcsname{\def\PYG@tc##1{\textcolor[rgb]{0.25,0.44,0.63}{##1}}}
\expandafter\def\csname PYG@tok@gp\endcsname{\let\PYG@bf=\textbf\def\PYG@tc##1{\textcolor[rgb]{0.78,0.36,0.04}{##1}}}
\expandafter\def\csname PYG@tok@sh\endcsname{\def\PYG@tc##1{\textcolor[rgb]{0.25,0.44,0.63}{##1}}}
\expandafter\def\csname PYG@tok@ow\endcsname{\let\PYG@bf=\textbf\def\PYG@tc##1{\textcolor[rgb]{0.00,0.44,0.13}{##1}}}
\expandafter\def\csname PYG@tok@sx\endcsname{\def\PYG@tc##1{\textcolor[rgb]{0.78,0.36,0.04}{##1}}}
\expandafter\def\csname PYG@tok@bp\endcsname{\def\PYG@tc##1{\textcolor[rgb]{0.00,0.44,0.13}{##1}}}
\expandafter\def\csname PYG@tok@c1\endcsname{\let\PYG@it=\textit\def\PYG@tc##1{\textcolor[rgb]{0.25,0.50,0.56}{##1}}}
\expandafter\def\csname PYG@tok@kc\endcsname{\let\PYG@bf=\textbf\def\PYG@tc##1{\textcolor[rgb]{0.00,0.44,0.13}{##1}}}
\expandafter\def\csname PYG@tok@c\endcsname{\let\PYG@it=\textit\def\PYG@tc##1{\textcolor[rgb]{0.25,0.50,0.56}{##1}}}
\expandafter\def\csname PYG@tok@mf\endcsname{\def\PYG@tc##1{\textcolor[rgb]{0.13,0.50,0.31}{##1}}}
\expandafter\def\csname PYG@tok@err\endcsname{\def\PYG@bc##1{\setlength{\fboxsep}{0pt}\fcolorbox[rgb]{1.00,0.00,0.00}{1,1,1}{\strut ##1}}}
\expandafter\def\csname PYG@tok@kd\endcsname{\let\PYG@bf=\textbf\def\PYG@tc##1{\textcolor[rgb]{0.00,0.44,0.13}{##1}}}
\expandafter\def\csname PYG@tok@ss\endcsname{\def\PYG@tc##1{\textcolor[rgb]{0.32,0.47,0.09}{##1}}}
\expandafter\def\csname PYG@tok@sr\endcsname{\def\PYG@tc##1{\textcolor[rgb]{0.14,0.33,0.53}{##1}}}
\expandafter\def\csname PYG@tok@mo\endcsname{\def\PYG@tc##1{\textcolor[rgb]{0.13,0.50,0.31}{##1}}}
\expandafter\def\csname PYG@tok@mi\endcsname{\def\PYG@tc##1{\textcolor[rgb]{0.13,0.50,0.31}{##1}}}
\expandafter\def\csname PYG@tok@kn\endcsname{\let\PYG@bf=\textbf\def\PYG@tc##1{\textcolor[rgb]{0.00,0.44,0.13}{##1}}}
\expandafter\def\csname PYG@tok@o\endcsname{\def\PYG@tc##1{\textcolor[rgb]{0.40,0.40,0.40}{##1}}}
\expandafter\def\csname PYG@tok@kr\endcsname{\let\PYG@bf=\textbf\def\PYG@tc##1{\textcolor[rgb]{0.00,0.44,0.13}{##1}}}
\expandafter\def\csname PYG@tok@s\endcsname{\def\PYG@tc##1{\textcolor[rgb]{0.25,0.44,0.63}{##1}}}
\expandafter\def\csname PYG@tok@kp\endcsname{\def\PYG@tc##1{\textcolor[rgb]{0.00,0.44,0.13}{##1}}}
\expandafter\def\csname PYG@tok@w\endcsname{\def\PYG@tc##1{\textcolor[rgb]{0.73,0.73,0.73}{##1}}}
\expandafter\def\csname PYG@tok@kt\endcsname{\def\PYG@tc##1{\textcolor[rgb]{0.56,0.13,0.00}{##1}}}
\expandafter\def\csname PYG@tok@sc\endcsname{\def\PYG@tc##1{\textcolor[rgb]{0.25,0.44,0.63}{##1}}}
\expandafter\def\csname PYG@tok@sb\endcsname{\def\PYG@tc##1{\textcolor[rgb]{0.25,0.44,0.63}{##1}}}
\expandafter\def\csname PYG@tok@k\endcsname{\let\PYG@bf=\textbf\def\PYG@tc##1{\textcolor[rgb]{0.00,0.44,0.13}{##1}}}
\expandafter\def\csname PYG@tok@se\endcsname{\let\PYG@bf=\textbf\def\PYG@tc##1{\textcolor[rgb]{0.25,0.44,0.63}{##1}}}
\expandafter\def\csname PYG@tok@sd\endcsname{\let\PYG@it=\textit\def\PYG@tc##1{\textcolor[rgb]{0.25,0.44,0.63}{##1}}}

\def\PYGZbs{\char`\\}
\def\PYGZus{\char`\_}
\def\PYGZob{\char`\{}
\def\PYGZcb{\char`\}}
\def\PYGZca{\char`\^}
\def\PYGZam{\char`\&}
\def\PYGZlt{\char`\<}
\def\PYGZgt{\char`\>}
\def\PYGZsh{\char`\#}
\def\PYGZpc{\char`\%}
\def\PYGZdl{\char`\$}
\def\PYGZhy{\char`\-}
\def\PYGZsq{\char`\'}
\def\PYGZdq{\char`\"}
\def\PYGZti{\char`\~}
% for compatibility with earlier versions
\def\PYGZat{@}
\def\PYGZlb{[}
\def\PYGZrb{]}
\makeatother

\begin{document}

\maketitle
\tableofcontents
\phantomsection\label{index::doc}


\textbf{sdocs is an akronym for small/simple dokumentation}

Contents:


\chapter{Overview}
\label{sdocs/overview:overview}\label{sdocs/overview:welcome-to-sdocs}\label{sdocs/overview::doc}
\begin{notice}{warning}{Warning:}
\begin{DUlineblock}{0em}
\item[] \# A collection of Howto's, Guides, Snippets collected from the web
\item[] \# Only for private use and there ist no warranty for correct information
\item[] \# You use it at your own risk, and all information is copyrighted by the owner
\item[] \# Most of this Source is written and collected by Jeffrey Scherling \footnote{
Have Fun!
}
\end{DUlineblock}
\end{notice}


\strong{See also:}


Google, for the source of these informations.



\begin{notice}{note}{Note:}
This document was generated on 2014-12-02 at 23:54.
\end{notice}


\chapter{sdocs}
\label{sdocs/index_sdocs::doc}\label{sdocs/index_sdocs:sdocs}
\# this is the index of sdocs


\section{Sphinx}
\label{sdocs/sphinx/sphinx:sphinx}\label{sdocs/sphinx/sphinx::doc}
\# a quick start guide

I Prerequisites
\begin{enumerate}
\item {} 
python

\item {} 
sphinx

\item {} 
webbrowser or pdfviewer

\end{enumerate}

II Build the documentation
\begin{enumerate}
\item {} 
enter sphinx-quickstart \# create the root directory of documentation

\item {} 
edit conf.py \# set the output to your needs

\item {} 
create your docu name.rst

\item {} 
add name.rst to index.rst

\item {} 
make html, latexpdf or linkcheck

\end{enumerate}

III Look at the Documentation
\begin{enumerate}
\item {} 
open index.html with your webbrowser

\item {} 
open projectname.pdf with your pdfviewer

\end{enumerate}

IV Markup
\begin{itemize}
\item {} 
markup \href{http://sphinx-doc.org/rest.html}{http://sphinx-doc.org/rest.html}

\item {} 
links \href{http://sphinx-doc.org/markup/inline.html\#ref-role}{http://sphinx-doc.org/markup/inline.html\#ref-role}

\item {} 
markup code \href{http://sphinx-doc.org/markup/code.html}{http://sphinx-doc.org/markup/code.html}

\item {} 
guide \href{http://docs.python-guide.org/en/latest/writing/documentation/}{http://docs.python-guide.org/en/latest/writing/documentation/}

\end{itemize}


\section{Webserver}
\label{sdocs/webserver/webserver:webserver}\label{sdocs/webserver/webserver::doc}
\# webserver and their configuration


\subsection{Nginx}
\label{sdocs/webserver/nginx/nginx:nginx}\label{sdocs/webserver/nginx/nginx::doc}\begin{itemize}
\item {} \begin{description}
\item[{nginx configuration}] \leavevmode\begin{itemize}
\item {} 
config files

\item {} 
websites configs

\end{itemize}

\end{description}

\end{itemize}

\# this is the main nginx configuration file {\hyperref[sdocs/webserver/nginx/nginx:nginx-conf]{nginx.conf}}
\phantomsection\label{sdocs/webserver/nginx/nginx:nginx-conf}
\begin{Verbatim}[commandchars=\\\{\}]
user	amorsql amorsql; \PYGZsh{} user group of processes
worker\PYGZus{}processes  2;

events \PYGZob{}
	worker\PYGZus{}connections  1024;
\PYGZcb{}

http \PYGZob{}
	include			mime.types;
	default\PYGZus{}type		application/octet\PYGZhy{}stream;
	gzip			on;
	gzip\PYGZus{}min\PYGZus{}length		5000;
	gzip\PYGZus{}buffers		4 8k;
	gzip\PYGZus{}types		text/plain text/css application/x\PYGZhy{}javascript text/xml application/xml application/xml+rss text/javascript;
	gzip\PYGZus{}proxied		any;
	gzip\PYGZus{}comp\PYGZus{}level		2;
	ignore\PYGZus{}invalid\PYGZus{}headers	on;
	include			sites\PYGZhy{}enabled/*;
	
	\PYGZsh{} test it
	sendfile		on;
\PYGZcb{}
\end{Verbatim}


\section{Ftpserver}
\label{sdocs/ftpserver/ftpserver:ftpserver}\label{sdocs/ftpserver/ftpserver::doc}
\# Ftpserver and their configuration


\subsection{Vsftpd}
\label{sdocs/ftpserver/vsftpd/vsftpd:vsftpd}\label{sdocs/ftpserver/vsftpd/vsftpd::doc}
\# configuration of vsftpd

\href{http://www.basicconfig.com/linuxnetwork/ftp\_server\#check-vsftpd}{http://www.basicconfig.com/linuxnetwork/ftp\_server\#check-vsftpd}
\href{http://wiki.ubuntuusers.de/vsftpd}{http://wiki.ubuntuusers.de/vsftpd}

\begin{Verbatim}[commandchars=\\\{\}]
\PYGZsh{} Example config file /etc/vsftpd.conf
\PYGZsh{}
\PYGZsh{} modified by Jeffrey Scherling for his needs 1 Dez 12
\PYGZsh{}
\PYGZsh{} The default compiled in settings are fairly paranoid. This sample file
\PYGZsh{} loosens things up a bit, to make the ftp daemon more usable.
\PYGZsh{} Please see vsftpd.conf.5 for all compiled in defaults.
\PYGZsh{}
\PYGZsh{} READ THIS: This example file is NOT an exhaustive list of vsftpd options.
\PYGZsh{} Please read the vsftpd.conf.5 manual page to get a full idea of vsftpd\PYGZsq{}s
\PYGZsh{} capabilities.
\PYGZsh{}
\PYGZsh{} Allow anonymous FTP? (Beware \PYGZhy{} allowed by default if you comment this out).
anonymous\PYGZus{}enable=NO
\PYGZsh{}
\PYGZsh{} Uncomment this to allow local users to log in.
local\PYGZus{}enable=YES
\PYGZsh{}
\PYGZsh{} Uncomment this to enable any form of FTP write command.
write\PYGZus{}enable=YES
\PYGZsh{}
\PYGZsh{} Default umask for local users is 077. You may wish to change this to 022,
\PYGZsh{} if your users expect that (022 is used by most other ftpd\PYGZsq{}s)
\PYGZsh{} default
local\PYGZus{}umask=022
\PYGZsh{}
\PYGZsh{}
\PYGZsh{}
\PYGZsh{} Uncomment this to allow the anonymous FTP user to upload files. This only has an effect if the above global write enable is activated. Also, you will obviously need to create a directory writable by the FTP user.
\PYGZsh{}anon\PYGZus{}upload\PYGZus{}enable=YES
\PYGZsh{}
\PYGZsh{} Uncomment this if you want the anonymous FTP user to be able to create
\PYGZsh{} new directories.
\PYGZsh{}anon\PYGZus{}mkdir\PYGZus{}write\PYGZus{}enable=YES
\PYGZsh{}
\PYGZsh{} Activate directory messages \PYGZhy{} messages given to remote users when they
\PYGZsh{} go into a certain directory.
dirmessage\PYGZus{}enable=YES
\PYGZsh{}
\PYGZsh{} Activate logging of uploads/downloads.
xferlog\PYGZus{}enable=YES
\PYGZsh{}
\PYGZsh{} Make sure PORT transfer connections originate from port 20 (ftp\PYGZhy{}data).
connect\PYGZus{}from\PYGZus{}port\PYGZus{}20=YES
\PYGZsh{}
\PYGZsh{} If you want, you can arrange for uploaded anonymous files to be owned by a different user. Note! Using \PYGZdq{}root\PYGZdq{} for uploaded files is not
\PYGZsh{} recommended!
\PYGZsh{}chown\PYGZus{}uploads=YES
\PYGZsh{}chown\PYGZus{}username=whoever
\PYGZsh{}
\PYGZsh{} You may override where the log file goes if you like. The default is shown
\PYGZsh{} below.
xferlog\PYGZus{}file=/var/log/vsftpd.log
\PYGZsh{}
\PYGZsh{} If you want, you can have your log file in standard ftpd xferlog format. Note that the default log file location is /var/log/xferlog in this case.
xferlog\PYGZus{}std\PYGZus{}format=YES
\PYGZsh{}
\PYGZsh{} You may change the default value for timing out an idle session.
\PYGZsh{}idle\PYGZus{}session\PYGZus{}timeout=600
\PYGZsh{}
\PYGZsh{} You may change the default value for timing out a data connection.
\PYGZsh{}data\PYGZus{}connection\PYGZus{}timeout=120
\PYGZsh{}
\PYGZsh{} It is recommended that you define on your system a unique user which the ftp server can use as a totally isolated and unprivileged user.
nopriv\PYGZus{}user=ftpsecure
\PYGZsh{}
\PYGZsh{} Enable this and the server will recognise asynchronous ABOR requests. Not recommended for security (the code is non\PYGZhy{}trivial). Not enabling it,
\PYGZsh{} however, may confuse older FTP clients.
\PYGZsh{}async\PYGZus{}abor\PYGZus{}enable=YES
\PYGZsh{}
\PYGZsh{} By default the server will pretend to allow ASCII mode but in fact ignore the request. Turn on the below options to have the server actually do ASCII
\PYGZsh{} mangling on files when in ASCII mode.
\PYGZsh{} Beware that on some FTP servers, ASCII support allows a denial of service
\PYGZsh{} attack (DoS) via the command \PYGZdq{}SIZE /big/file\PYGZdq{} in ASCII mode. vsftpd
\PYGZsh{} predicted this attack and has always been safe, reporting the size of the
\PYGZsh{} raw file.
\PYGZsh{} ASCII mangling is a horrible feature of the protocol.
\PYGZsh{}ascii\PYGZus{}upload\PYGZus{}enable=YES
\PYGZsh{}ascii\PYGZus{}download\PYGZus{}enable=YES
\PYGZsh{}
\PYGZsh{} You may fully customise the login banner string:
\PYGZsh{}ftpd\PYGZus{}banner=\PYGZdq{}      \PYGZus{}\PYGZus{}\PYGZus{}\PYGZus{}\PYGZus{}\PYGZus{}\PYGZus{}\PYGZus{}\PYGZus{}\PYGZus{}\PYGZus{}\PYGZus{}\PYGZus{}\PYGZus{}\PYGZus{}\PYGZus{}\PYGZus{}\PYGZus{}\PYGZus{}\PYGZus{}\PYGZus{}\PYGZus{}\PYGZus{}\PYGZus{}\PYGZus{}\PYGZus{}\PYGZus{}\PYGZus{}\PYGZus{}\PYGZus{}\PYGZus{}\PYGZus{}\PYGZus{}\PYGZus{}\PYGZus{}\PYGZus{}\PYGZus{}\PYGZus{}\PYGZus{}\PYGZus{}\PYGZus{}\PYGZus{}\PYGZus{}\PYGZus{}Welcome to my ftp Site!\PYGZus{}\PYGZus{}\PYGZus{}\PYGZus{}\PYGZus{}\PYGZdq{}
\PYGZsh{}ftpd\PYGZus{}banner=\PYGZdq{}             ooo                 ooo\PYGZdq{}
\PYGZsh{}ftpd\PYGZus{}banner=\PYGZdq{}             oo                  oo\PYGZdq{}
\PYGZsh{}ftpd\PYGZus{}banner=\PYGZdq{}    ooooo   oo   oooo    oooo   oo  ooo ooo        ooo oooo   ooo ooo   oooo\PYGZdq{}
\PYGZsh{}ftpd\PYGZus{}banner=\PYGZdq{}   oo      oo      oo  oo      oo oo    oo   oo   oo     oo   ooo  oo oo  oo\PYGZdq{}
\PYGZsh{}ftpd\PYGZus{}banner=\PYGZdq{}   ooo    oo   ooooo  oo      oooo      oo oooo oo   ooooo   oo      oooooo\PYGZdq{}
\PYGZsh{}ftpd\PYGZus{}banner=\PYGZdq{}    oo   oo  oo  oo  oo      oo oo      ooo  ooo   oo  oo   oo      oo\PYGZdq{}
\PYGZsh{}ftpd\PYGZus{}banner=\PYGZdq{}ooooo  oooo  oooooo  oooo  ooo  ooo    oo    oo    oooooo  oo       ooooo\PYGZdq{}
\PYGZsh{}ftpd\PYGZus{}banner=\PYGZdq{}\PYGZus{}\PYGZus{}\PYGZus{}\PYGZus{}\PYGZus{}\PYGZus{}\PYGZus{}\PYGZus{}\PYGZus{}\PYGZus{}\PYGZus{}\PYGZus{}\PYGZus{}\PYGZus{}\PYGZus{}\PYGZus{}\PYGZus{}\PYGZus{}\PYGZus{}\PYGZus{}\PYGZus{}\PYGZus{}\PYGZus{}\PYGZus{}\PYGZus{}\PYGZus{}\PYGZus{}\PYGZus{}\PYGZus{}\PYGZus{}\PYGZus{}\PYGZus{}\PYGZus{}\PYGZus{}\PYGZus{}\PYGZus{}\PYGZus{}\PYGZus{}\PYGZus{}\PYGZus{}\PYGZus{}\PYGZus{}\PYGZus{}\PYGZus{}\PYGZus{}\PYGZus{}\PYGZus{}\PYGZus{}\PYGZus{}\PYGZus{}\PYGZus{}\PYGZus{}\PYGZus{}\PYGZus{}\PYGZus{}\PYGZus{}\PYGZus{}\PYGZus{}\PYGZus{}\PYGZus{}\PYGZus{}Jeffrey\PYGZus{}\PYGZus{}\PYGZus{}\PYGZus{}\PYGZus{}\PYGZdq{}
\PYGZsh{}
\PYGZsh{} customize your login
banner\PYGZus{}file=/etc/vsftpd.banner\PYGZus{}file
\PYGZsh{}
\PYGZsh{} You may specify a file of disallowed anonymous e\PYGZhy{}mail addresses. Apparently
\PYGZsh{} useful for combatting certain DoS attacks.
\PYGZsh{}deny\PYGZus{}email\PYGZus{}enable=YES
\PYGZsh{} (default follows)
\PYGZsh{}banned\PYGZus{}email\PYGZus{}file=/etc/vsftpd.banned\PYGZus{}emails
\PYGZsh{}
\PYGZsh{} You may specify an explicit list of local users to chroot() to their home
\PYGZsh{} directory. If chroot\PYGZus{}local\PYGZus{}user is YES, then this list becomes a list of
\PYGZsh{} users to NOT chroot().
\PYGZsh{} (Warning! chroot\PYGZsq{}ing can be very dangerous. If using chroot, make sure that
\PYGZsh{} the user does not have write access to the top level directory within the
\PYGZsh{} chroot)
\PYGZsh{} dangerous don\PYGZsq{}t use it
chroot\PYGZus{}local\PYGZus{}user=NO
chroot\PYGZus{}list\PYGZus{}enable=YES
passwd\PYGZus{}chroot\PYGZus{}enable=YES
chroot\PYGZus{}list\PYGZus{}file=/etc/vsftpd.chroot\PYGZus{}list
\PYGZsh{}
\PYGZsh{} You may activate the \PYGZdq{}\PYGZhy{}R\PYGZdq{} option to the builtin ls. This is disabled by
\PYGZsh{} default to avoid remote users being able to cause excessive I/O on large
\PYGZsh{} sites. However, some broken FTP clients such as \PYGZdq{}ncftp\PYGZdq{} and \PYGZdq{}mirror\PYGZdq{} assume
\PYGZsh{} the presence of the \PYGZdq{}\PYGZhy{}R\PYGZdq{} option, so there is a strong case for enabling it.
ls\PYGZus{}recurse\PYGZus{}enable=YES
\PYGZsh{}
\PYGZsh{} When \PYGZdq{}listen\PYGZdq{} directive is enabled, vsftpd runs in standalone mode (rather
\PYGZsh{} than from inetd) and listens on IPv4 sockets. To use vsftpd in standalone
\PYGZsh{} mode rather than with inetd, change the line below to \PYGZsq{}listen=YES\PYGZsq{}
\PYGZsh{} This directive cannot be used in conjunction with the listen\PYGZus{}ipv6 directive.
listen=NO
\PYGZsh{}
\PYGZsh{} This directive enables listening on IPv6 sockets. To listen on IPv4 and IPv6
\PYGZsh{} sockets, you must run two copies of vsftpd with two configuration files.
\PYGZsh{} Make sure, that one of the listen options is commented !!
\PYGZsh{}listen\PYGZus{}ipv6=YES
\PYGZsh{}
\PYGZsh{} adds by jeff
\PYGZsh{}
\PYGZsh{} allow write with chroot
allow\PYGZus{}writeable\PYGZus{}chroot=YES
\PYGZsh{}
\PYGZsh{} access to only this users
userlist\PYGZus{}deny=NO
userlist\PYGZus{}enable=YES
userlist\PYGZus{}file=/etc/vsftpd.user\PYGZus{}list
\PYGZsh{}
\PYGZsh{} ssl
\PYGZsh{}ssl\PYGZus{}enable=NO
\PYGZsh{}ssl\PYGZus{}sslv2=YES
\PYGZsh{}create ssl \PYGZhy{} Zertifikat for ssl using
\PYGZsh{}openssl req \PYGZhy{}x509 \PYGZhy{}nodes \PYGZhy{}days 365 \PYGZhy{}newkey rsa:1024  \PYGZhy{}keyout /etc/ssl/private/vsftpd.pem \PYGZhy{}out /etc/ssl/private/vsftpd.pem
\PYGZsh{}
\PYGZsh{} guests remapping all non annonymus to this login
\PYGZsh{}guest\PYGZus{}enable=YES
\PYGZsh{}guest\PYGZus{}username=ftpuser
\PYGZsh{}
max\PYGZus{}clients=3
max\PYGZus{}per\PYGZus{}ip=2
\PYGZsh{}
tilde\PYGZus{}user\PYGZus{}enable=YES
\PYGZsh{}
\PYGZsh{} hide and deny files and directories
hide\PYGZus{}file=\PYGZob{}/,/media,/gamma\PYGZus{}sftp\PYGZcb{} 
deny\PYGZus{}file=\PYGZob{}/,/media,/gamma\PYGZus{}sftp\PYGZcb{} 
\PYGZsh{}
\PYGZsh{} ftp commands to deny 
\PYGZsh{} deny change to the parent of the current working directory.
\PYGZsh{}cmds\PYGZus{}denied=XCUP
\PYGZsh{}
\PYGZsh{} set the default mmask
\PYGZsh{}file\PYGZus{}open\PYGZus{}mode=0777
\PYGZsh{}
\PYGZsh{}
\PYGZsh{}
\PYGZsh{}
\end{Verbatim}

\# restart the process

\href{mailto:root@gamma}{root@gamma}:\textasciitilde{}\# /etc/rc.d/rc.inetd restart
Starting Internet super-server daemon:  /usr/sbin/inetd

\# customize the login

\begin{Verbatim}[commandchars=\\\{\}]
      \PYGZus{}\PYGZus{}\PYGZus{}\PYGZus{}\PYGZus{}\PYGZus{}\PYGZus{}\PYGZus{}\PYGZus{}\PYGZus{}\PYGZus{}\PYGZus{}\PYGZus{}\PYGZus{}\PYGZus{}\PYGZus{}\PYGZus{}\PYGZus{}\PYGZus{}\PYGZus{}\PYGZus{}\PYGZus{}\PYGZus{}\PYGZus{}\PYGZus{}\PYGZus{}\PYGZus{}\PYGZus{}\PYGZus{}\PYGZus{}\PYGZus{}\PYGZus{}\PYGZus{}\PYGZus{}\PYGZus{}\PYGZus{}\PYGZus{}\PYGZus{}\PYGZus{}\PYGZus{}\PYGZus{}\PYGZus{}\PYGZus{}\PYGZus{}Welcome to my ftp Site!\PYGZus{}\PYGZus{}\PYGZus{}\PYGZus{}\PYGZus{}
             ooo                 ooo
             oo                  oo
    ooooo   oo   oooo    oooo   oo  ooo ooo        ooo oooo   ooo ooo   oooo
   oo      oo      oo  oo      oo oo    oo   oo   oo     oo   ooo  oo oo  oo
   ooo    oo   ooooo  oo      oooo      oo oooo oo   ooooo   oo      oooooo
    oo   oo  oo  oo  oo      oo oo      ooo  ooo   oo  oo   oo      oo
ooooo  oooo  oooooo  oooo  ooo  ooo    oo    oo    oooooo  oo       ooooo
\PYGZus{}\PYGZus{}\PYGZus{}\PYGZus{}\PYGZus{}\PYGZus{}\PYGZus{}\PYGZus{}\PYGZus{}\PYGZus{}\PYGZus{}\PYGZus{}\PYGZus{}\PYGZus{}\PYGZus{}\PYGZus{}\PYGZus{}\PYGZus{}\PYGZus{}\PYGZus{}\PYGZus{}\PYGZus{}\PYGZus{}\PYGZus{}\PYGZus{}\PYGZus{}\PYGZus{}\PYGZus{}\PYGZus{}\PYGZus{}\PYGZus{}\PYGZus{}\PYGZus{}\PYGZus{}\PYGZus{}\PYGZus{}\PYGZus{}\PYGZus{}\PYGZus{}\PYGZus{}\PYGZus{}\PYGZus{}\PYGZus{}\PYGZus{}\PYGZus{}\PYGZus{}\PYGZus{}\PYGZus{}\PYGZus{}\PYGZus{}\PYGZus{}\PYGZus{}\PYGZus{}\PYGZus{}\PYGZus{}\PYGZus{}\PYGZus{}\PYGZus{}\PYGZus{}\PYGZus{}\PYGZus{}Jeffrey\PYGZus{}\PYGZus{}\PYGZus{}\PYGZus{}\PYGZus{}
\end{Verbatim}

\# user who are permitted to login

\begin{Verbatim}[commandchars=\\\{\}]
\PYG{n}{jeffrey}
\PYG{n}{slackuser}
\PYG{n}{vitalij}
\PYG{n}{achim}
\PYG{n}{jens}
\end{Verbatim}

\# user who login with chroot in a jail

\begin{Verbatim}[commandchars=\\\{\}]
\PYG{n}{jeffrey}
\PYG{n}{slackuser}
\PYG{n}{vitalij}
\PYG{n}{achim}
\PYG{n}{jens}
\end{Verbatim}


\section{Databases}
\label{sdocs/databases/databases::doc}\label{sdocs/databases/databases:databases}
\# configuration of different databeses


\subsection{MariaDB}
\label{sdocs/databases/mariadb/mariadb:mariadb}\label{sdocs/databases/mariadb/mariadb::doc}
\# Howto Setup MariaDB

I Prerequisites
\begin{enumerate}
\item {} 
mariadb-5.5.40-x86\_64-2.txz

\item {} 
privileged user only for sql-data, mostly mysql

\end{enumerate}

II Installation for Slackware
\begin{enumerate}
\item {} 
use slackpkg

\item {} 
use sbopkg

\item {} 
use SlackBuild scripts

\end{enumerate}

III Configuration

Howto: \href{http://docs.slackware.com/howtos:databases:install\_mariadb\_on\_slackware}{http://docs.slackware.com/howtos:databases:install\_mariadb\_on\_slackware}
\begin{enumerate}
\item {} 
mysql\_install\_db --user=mysql

\item {} 
chown -R mysql.mysql /var/lib/mysql

\item {} 
chown 755 /etc/rc.d/rc.mysqld

\item {} 
/etc/rc.d/rc.mysqld start

\item {} 
mysqladmin -u root password `NEW\_PASSWORD'

\item {} 
use phpmyadmin to manage databases

\end{enumerate}

\begin{Verbatim}[commandchars=\\\{\}]
root@gamma:\PYGZti{}\PYGZsh{} mysql\PYGZus{}install\PYGZus{}db \PYGZhy{}\PYGZhy{}user=mysql
Installing MariaDB/MySQL system tables in \PYGZsq{}/var/lib/mysql\PYGZsq{} ...
OK
Filling help tables...
OK

To start mysqld at boot time you have to copy
support\PYGZhy{}files/mysql.server to the right place for your system

PLEASE REMEMBER TO SET A PASSWORD FOR THE MariaDB root USER !
To do so, start the server, then issue the following commands:

\PYGZsq{}/usr/bin/mysqladmin\PYGZsq{} \PYGZhy{}u root password \PYGZsq{}new\PYGZhy{}password\PYGZsq{}
\PYGZsq{}/usr/bin/mysqladmin\PYGZsq{} \PYGZhy{}u root \PYGZhy{}h gamma password \PYGZsq{}new\PYGZhy{}password\PYGZsq{}

Alternatively you can run:
\PYGZsq{}/usr/bin/mysql\PYGZus{}secure\PYGZus{}installation\PYGZsq{}

which will also give you the option of removing the test
databases and anonymous user created by default.  This is
strongly recommended for production servers.

See the MariaDB Knowledgebase at http://mariadb.com/kb or the
MySQL manual for more instructions.

You can start the MariaDB daemon with:
cd \PYGZsq{}/usr\PYGZsq{} ; /usr/bin/mysqld\PYGZus{}safe \PYGZhy{}\PYGZhy{}datadir=\PYGZsq{}/var/lib/mysql\PYGZsq{}

You can test the MariaDB daemon with mysql\PYGZhy{}test\PYGZhy{}run.pl
cd \PYGZsq{}/usr/mysql\PYGZhy{}test\PYGZsq{} ; perl mysql\PYGZhy{}test\PYGZhy{}run.pl

Please report any problems at http://mariadb.org/jira

The latest information about MariaDB is available at http://mariadb.org/.
You can find additional information about the MySQL part at:
http://dev.mysql.com
Support MariaDB development by buying support/new features from
SkySQL Ab. You can contact us about this at sales@skysql.com.
Alternatively consider joining our community based development effort:
http://mariadb.com/kb/en/contributing\PYGZhy{}to\PYGZhy{}the\PYGZhy{}mariadb\PYGZhy{}project/

root@gamma:\PYGZti{}\PYGZsh{}
\end{Verbatim}


\subsection{PostgreSQL}
\label{sdocs/databases/postgresql/postgresql:postgresql}\label{sdocs/databases/postgresql/postgresql::doc}
\# Howto Setup Postgresql

I Prerequisites
\begin{enumerate}
\item {} 
postgresql-5.1.tar.gz

\item {} 
privileged user only for sql-data

\end{enumerate}

II Installation for Slackware
\begin{enumerate}
\item {} 
use slackpkg

\item {} 
use sbopkg

\item {} 
use SlackBuild scripts

\end{enumerate}

III Configuration
\begin{enumerate}
\item {} 
create database

\end{enumerate}

\begin{Verbatim}[commandchars=\\\{\}]
root@gamma:\PYGZti{}\PYGZsh{} su amorsql \PYGZhy{}c \PYGZdq{}initdb \PYGZhy{}D /var/lib/pgsql/9.3/data \PYGZhy{}\PYGZhy{}locale=en\PYGZus{}US.UTF\PYGZhy{}8 \PYGZhy{}A md5 \PYGZhy{}W\PYGZdq{} 

could not change directory to \PYGZdq{}/root\PYGZdq{}: Permission denied
The files belonging to this database system will be owned by user \PYGZdq{}amorsql\PYGZdq{}.
This user must also own the server process.

The database cluster will be initialized with locale \PYGZdq{}en\PYGZus{}US.UTF\PYGZhy{}8\PYGZdq{}.
The default database encoding has accordingly been set to \PYGZdq{}UTF8\PYGZdq{}.
The default text search configuration will be set to \PYGZdq{}english\PYGZdq{}.

Data page checksums are disabled.

fixing permissions on existing directory /var/lib/pgsql/9.3/data ... ok
creating subdirectories ... ok
selecting default max\PYGZus{}connections ... 100
selecting default shared\PYGZus{}buffers ... 128MB
creating configuration files ... ok
creating template1 database in /var/lib/pgsql/9.3/data/base/1 ... ok
initializing pg\PYGZus{}authid ... ok
Enter new superuser password: 
Enter it again: 
setting password ... ok
initializing dependencies ... ok
creating system views ... ok
loading system objects\PYGZsq{} descriptions ... ok
creating collations ... ok
creating conversions ... ok
creating dictionaries ... ok
setting privileges on built\PYGZhy{}in objects ... ok
creating information schema ... ok
loading PL/pgSQL server\PYGZhy{}side language ... ok
vacuuming database template1 ... ok
copying template1 to template0 ... ok
copying template1 to postgres ... ok
syncing data to disk ... ok

Success. You can now start the database server using:

postgres \PYGZhy{}D /var/lib/pgsql/9.3/data
or
pg\PYGZus{}ctl \PYGZhy{}D /var/lib/pgsql/9.3/data \PYGZhy{}l logfile start

root@gamma:\PYGZti{}\PYGZsh{}
\end{Verbatim}
\begin{enumerate}
\setcounter{enumi}{1}
\item {} 
start the database

\item {} 
change user and group to what you set, edit rc.postgresql

\end{enumerate}

\begin{Verbatim}[commandchars=\\\{\}]
root@gamma:\PYGZti{}\PYGZsh{} /etc/rc.d/rc.postgresql start
Starting PostgreSQL
waiting for server to start.... done
server started
root@gamma:\PYGZti{}\PYGZsh{}
\end{Verbatim}
\begin{enumerate}
\setcounter{enumi}{3}
\item {} 
manage databases with phpPgAdmin

\end{enumerate}


\subsection{PHPmyadmin}
\label{sdocs/databases/phpmyadmin/phpmyadmin:phpmyadmin}\label{sdocs/databases/phpmyadmin/phpmyadmin::doc}
\# Howto Setup PHPmyadmin

I Prerequisites
\begin{enumerate}
\item {} 
phpmyadmin-4.2.11-noarch-1js.txz

\end{enumerate}

II Installation for Slackware
\begin{enumerate}
\item {} 
use slackpkg

\item {} 
use sbopkg

\item {} 
use SlackBuild scripts

\end{enumerate}

III Configuration

Setup: \href{http://wiki.phpmyadmin.net/pma/Setup}{http://wiki.phpmyadmin.net/pma/Setup}
Quick-Install: \href{http://wiki.phpmyadmin.net/pma/Quick\_Install}{http://wiki.phpmyadmin.net/pma/Quick\_Install}
\begin{enumerate}
\item {} 
cd phpMyAdmin

\item {} 
mkdir config                        \# create directory for saving

\item {} 
chmod o+rw config                   \# give it world writable permissions

\item {} 
cp config.inc.php config/           \# copy current configuration for editing

\item {} 
chmod o+w config/config.inc.php     \# give it world writable permissions

\item {} 
open \href{http://phpmyadmin/setup/}{http://phpmyadmin/setup/}       \# see nginx config

\item {} 
mv config/config.inc.php .          \# move file to current directory

\item {} 
chmod o-rw config.inc.php           \# remove world read and write permissions

\item {} 
open \href{http://phpmyadmin/}{http://phpmyadmin/}

\item {} 
login with root login and password

\item {} 
start adminstration

\end{enumerate}


\subsection{phpPgAdmin}
\label{sdocs/databases/phpPgAdmin/phpPgAdmin::doc}\label{sdocs/databases/phpPgAdmin/phpPgAdmin:phppgadmin}
\# Howto Setup phpPgAdmin

I Prerequisites
\begin{enumerate}
\item {} 
phpPgAdmin-5.1-noarch-1js

\end{enumerate}

II Installation for Slackware
\begin{enumerate}
\item {} 
use slackpkg

\item {} 
use sbopkg

\item {} 
use SlackBuild scripts

\end{enumerate}

III Configuration

Setup: \href{http://phppgadmin.sourceforge.net/doku.php}{http://phppgadmin.sourceforge.net/doku.php}


\section{Drupal}
\label{sdocs/drupal/drupal:drupal}\label{sdocs/drupal/drupal::doc}
setup drupal: \href{https://www.drupal.org/start}{https://www.drupal.org/start}
drupal with nginx: \href{http://wiki.nginx.org/Drupal}{http://wiki.nginx.org/Drupal}


\section{Django}
\label{sdocs/django/django::doc}\label{sdocs/django/django:django}
\# Howto Setup a Nginx, Django, Postgresql, Gunicorn deployment

I Prerequisites
\begin{enumerate}
\item {} 
server: nginx

\item {} 
framework: django

\item {} 
databank: postgresql (mariadb)

\item {} 
python-translator: gunicorn (uwsgi)

\end{enumerate}

Addons
a. virtual environments: virtualenv
b. documentation: sphinx
c. maybe pip for easy installation

II Installation for Slackware
\begin{enumerate}
\item {} 
use slackpkg

\item {} 
use sbopkg

\item {} 
use SlackBuild scripts

\end{enumerate}

III Configuration


\subsection{Django-Setup}
\label{sdocs/django/django-setup/django-setup:django-setup}\label{sdocs/django/django-setup/django-setup::doc}
\# Howto Setup Django

I Prerequisites
\begin{enumerate}
\item {} 
webserver environment with php and database

\item {} 
framework: django

\item {} 
python-translator: gunicorn (uwsgi)

\end{enumerate}

Addons
a. virtual environments: virtualenv

II Installation for Slackware
\begin{enumerate}
\item {} 
use slackpkg

\item {} 
use sbopkg

\item {} 
use SlackBuild scripts

\end{enumerate}

III Configuration


\subsection{Gunicorn}
\label{sdocs/django/gunicorn/gunicorn::doc}\label{sdocs/django/gunicorn/gunicorn:gunicorn}
I Build your First App with Gunicorn

\begin{Verbatim}[commandchars=\\\{\}]
  \PYGZdl{} sudo pip install virtualenv
  \PYGZdl{} mkdir \PYGZti{}/environments/
  \PYGZdl{} virtualenv \PYGZti{}/environments/tutorial/
  \PYGZdl{} cd \PYGZti{}/environments/tutorial/
  \PYGZdl{} ls
  bin  include  lib
  \PYGZdl{} source bin/activate
  (tutorial) \PYGZdl{} pip install gunicorn
  (tutorial) \PYGZdl{} mkdir myapp
  (tutorial) \PYGZdl{} cd myapp/
  (tutorial) \PYGZdl{} vi myapp.py
  (tutorial) \PYGZdl{} cat myapp.py

  def app(environ, start\PYGZus{}response):
      data = \PYGZdq{}Hello, World!\PYGZbs{}n\PYGZdq{}
      start\PYGZus{}response(\PYGZdq{}200 OK\PYGZdq{}, [
      (\PYGZdq{}Content\PYGZhy{}Type\PYGZdq{}, \PYGZdq{}text/plain\PYGZdq{}),
      (\PYGZdq{}Content\PYGZhy{}Length\PYGZdq{}, str(len(data)))
      ])
      return iter([data])

  (tutorial) \PYGZdl{} ../bin/gunicorn \PYGZhy{}w 4 myapp:app

2010\PYGZhy{}06\PYGZhy{}05 23:27:07 [16800] [INFO] Arbiter booted 
2010\PYGZhy{}06\PYGZhy{}05 23:27:07 [16800] [INFO] Listening at: http://127.0.0.1:8000 
2010\PYGZhy{}06\PYGZhy{}05 23:27:07 [16801] [INFO] Worker spawned (pid: 16801) 
2010\PYGZhy{}06\PYGZhy{}05 23:27:07 [16802] [INFO] Worker spawned (pid: 16802) 
2010\PYGZhy{}06\PYGZhy{}05 23:27:07 [16803] [INFO] Worker spawned (pid: 16803) 
2010\PYGZhy{}06\PYGZhy{}05 23:27:07 [16804] [INFO] Worker spawned (pid: 16804)
\end{Verbatim}


\section{Github}
\label{sdocs/github/github:github}\label{sdocs/github/github::doc}
\# Howto Setup Github

I Prerequisites
\begin{enumerate}
\item {} 
git

\item {} 
ssh

\item {} 
login on github

\end{enumerate}

II Installation for Slackware
\begin{enumerate}
\item {} 
use slackpkg

\item {} 
use sbopkg

\item {} 
use SlackBuild scripts

\end{enumerate}

III Configuration

\# quick setup

create ssh key
add ssh key
create a repo in github

initialisize repo and push it to remote repo with same name
touch README.md
git init
git add README.md
git commit -m ``first commit''
git remote add origin \href{mailto:git@github.com}{git@github.com}:liodopolus/scriptcol-current.git
git push -u origin master
(git remote add upstream \href{mailto:git@github.com}{git@github.com}:liodopolus/scriptcol-current.git)

\# simple howto

\href{http://githowto.com/setup}{http://githowto.com/setup}

\# setup github
\href{https://help.github.com/articles/set-up-git/}{https://help.github.com/articles/set-up-git/}

see too man git or man gittutorial:

\begin{Verbatim}[commandchars=\\\{\}]
git config \PYGZhy{}\PYGZhy{}global user.name \PYGZdq{}YOUR NAME\PYGZdq{}
git config \PYGZhy{}\PYGZhy{}global user.email \PYGZdq{}YOUR EMAIL ADDRESS\PYGZdq{}
\end{Verbatim}

\# create a repo

\begin{DUlineblock}{0em}
\item[] \href{https://help.github.com/articles/create-a-repo/}{https://help.github.com/articles/create-a-repo/}
\item[] \href{https://help.github.com/articles/fork-a-repo/}{https://help.github.com/articles/fork-a-repo/}
\item[] \href{https://help.github.com/articles/be-social/}{https://help.github.com/articles/be-social/}
\end{DUlineblock}

\# git docs

\href{http://git-scm.com/}{http://git-scm.com/}

\# sync a fork

\href{https://help.github.com/articles/syncing-a-fork/}{https://help.github.com/articles/syncing-a-fork/}

\# push a remote
\begin{quote}

gitstatus
git status

first add
git add *

then commit
git commit -m ``Kommentar''

then push
git push
\end{quote}

\href{https://help.github.com/articles/pushing-to-a-remote/}{https://help.github.com/articles/pushing-to-a-remote/}


\section{Backup}
\label{sdocs/backup/backup:backup}\label{sdocs/backup/backup::doc}
\# cool backup software

\href{http://www.cyberciti.biz/open-source/awesome-backup-software-for-linux-unix-osx-windows-systems/}{http://www.cyberciti.biz/open-source/awesome-backup-software-for-linux-unix-osx-windows-systems/}


\subsection{Rsnapshot}
\label{sdocs/backup/rsnapshot/rsnapshot::doc}\label{sdocs/backup/rsnapshot/rsnapshot:rsnapshot}
\href{http://www.rsnapshot.org/}{http://www.rsnapshot.org/}


\section{Readthedocs}
\label{sdocs/readthedocs/readthedocs:readthedocs}\label{sdocs/readthedocs/readthedocs::doc}
\# a cool site for documentations

\href{https://readthedocs.org/}{https://readthedocs.org/}


\chapter{ToDo}
\label{index:todo}\begin{itemize}
\item {} 
setup git (in progress)

\item {} 
setup readthedocs (in progress)

\item {} 
setup ftpserver

\item {} 
send vitali and achim link to photos

\item {} 
setup phpPgAmdin

\item {} 
setup postgresql

\item {} 
setup gunicorn

\item {} 
setup django

\item {} 
setup mini server

\item {} 
setup mutt and sendmail

\item {} 
read root mail from rsnapshot

\end{itemize}


\chapter{Done}
\label{index:done}\begin{itemize}
\item {} 
setup rsnapshot (finished)

\end{itemize}


\chapter{Indices and tables}
\label{index:indices-and-tables}\begin{itemize}
\item {} 
\emph{genindex}

\item {} 
\emph{modindex}

\item {} 
\emph{search}

\end{itemize}



\renewcommand{\indexname}{Index}
\printindex
\end{document}
